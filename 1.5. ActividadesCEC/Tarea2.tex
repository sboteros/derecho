% Options for packages loaded elsewhere
\PassOptionsToPackage{unicode}{hyperref}
\PassOptionsToPackage{hyphens}{url}
%
\documentclass[
  spanish,
]{article}
\usepackage{lmodern}
\usepackage{amssymb,amsmath}
\usepackage{ifxetex,ifluatex}
\ifnum 0\ifxetex 1\fi\ifluatex 1\fi=0 % if pdftex
  \usepackage[T1]{fontenc}
  \usepackage[utf8]{inputenc}
  \usepackage{textcomp} % provide euro and other symbols
\else % if luatex or xetex
  \usepackage{unicode-math}
  \defaultfontfeatures{Scale=MatchLowercase}
  \defaultfontfeatures[\rmfamily]{Ligatures=TeX,Scale=1}
\fi
% Use upquote if available, for straight quotes in verbatim environments
\IfFileExists{upquote.sty}{\usepackage{upquote}}{}
\IfFileExists{microtype.sty}{% use microtype if available
  \usepackage[]{microtype}
  \UseMicrotypeSet[protrusion]{basicmath} % disable protrusion for tt fonts
}{}
\makeatletter
\@ifundefined{KOMAClassName}{% if non-KOMA class
  \IfFileExists{parskip.sty}{%
    \usepackage{parskip}
  }{% else
    \setlength{\parindent}{0pt}
    \setlength{\parskip}{6pt plus 2pt minus 1pt}}
}{% if KOMA class
  \KOMAoptions{parskip=half}}
\makeatother
\usepackage{xcolor}
\IfFileExists{xurl.sty}{\usepackage{xurl}}{} % add URL line breaks if available
\IfFileExists{bookmark.sty}{\usepackage{bookmark}}{\usepackage{hyperref}}
\hypersetup{
  pdftitle={La modificación del artículo 1 constitucional},
  pdfauthor={Santiago Botero Sierra},
  hidelinks,
  pdfcreator={LaTeX via pandoc}}
\urlstyle{same} % disable monospaced font for URLs
\usepackage[margin=1in]{geometry}
\usepackage{graphicx,grffile}
\makeatletter
\def\maxwidth{\ifdim\Gin@nat@width>\linewidth\linewidth\else\Gin@nat@width\fi}
\def\maxheight{\ifdim\Gin@nat@height>\textheight\textheight\else\Gin@nat@height\fi}
\makeatother
% Scale images if necessary, so that they will not overflow the page
% margins by default, and it is still possible to overwrite the defaults
% using explicit options in \includegraphics[width, height, ...]{}
\setkeys{Gin}{width=\maxwidth,height=\maxheight,keepaspectratio}
% Set default figure placement to htbp
\makeatletter
\def\fps@figure{htbp}
\makeatother
\setlength{\emergencystretch}{3em} % prevent overfull lines
\providecommand{\tightlist}{%
  \setlength{\itemsep}{0pt}\setlength{\parskip}{0pt}}
\setcounter{secnumdepth}{-\maxdimen} % remove section numbering
\usepackage{fixltx2e}
\usepackage{booktabs}
\usepackage{longtable}
\usepackage{array}
\usepackage{multirow}
\usepackage{wrapfig}
\usepackage{float}
\usepackage{colortbl}
\usepackage{pdflscape}
\usepackage{tabu}
\usepackage{threeparttable}
\usepackage{threeparttablex}
\usepackage[normalem]{ulem}
\usepackage{makecell}
\usepackage{xcolor}
\ifxetex
  % Load polyglossia as late as possible: uses bidi with RTL langages (e.g. Hebrew, Arabic)
  \usepackage{polyglossia}
  \setmainlanguage[]{spanish}
\else
  \usepackage[shorthands=off,main=spanish]{babel}
\fi

\title{La modificación del artículo 1 constitucional}
\usepackage{etoolbox}
\makeatletter
\providecommand{\subtitle}[1]{% add subtitle to \maketitle
  \apptocmd{\@title}{\par {\large #1 \par}}{}{}
}
\makeatother
\subtitle{El reconocimiento de los derechos humanos en México}
\author{Santiago Botero Sierra}
\date{4 de febrero de 2020}

\begin{document}
\maketitle

¡hOLA!

\begin{landscape}
\begin{ThreePartTable}
\begin{TableNotes}
\item \textit{Notas:} 
\item[1] Constitución Federal de los Estados Unidos Mexicanos, 1857.
\item[2] Constitución Política de los Estados Unidos Mexicanos.
\item[3] DECRETO por el que se aprueba el diverso por el que se adicionan un segundo y tercer párrafos al artículo 1o., se reforma el artículo 2o., se deroga el párrafo primero del artículo 4o.; y se adicionan un sexto párrafo al artículo 18, y un último párrafo a la fracción tercera del artículo 115 de la Constitución Política de los Estados Unidos Mexicanos, publicado en el DOF el 14 de agosto de 2001.
\item[4] DECRETO por el que se reforma el Artículo 1o., Párrafo Tercero de la Constitución Política de los Estados Unidos Mexicanos, publicado en el DOF el 4 de diciembre de 2006.
\item[5] DECRETO por el que se modifica la denominación del Capítulo 1 del Título Primero y reforma diversos artículos de la Constitución Política de los Estados Unidos Mexicanos, publicado en el DOF el 10 de junio de 2011.
\end{TableNotes}
\begin{longtable}[t]{>{\raggedright\arraybackslash}p{0.2 * linewidth}|>{\raggedright\arraybackslash}p{0.2 * linewidth}|>{\raggedright\arraybackslash}p{0.2 * linewidth}|>{\raggedright\arraybackslash}p{0.2 * linewidth}|>{\raggedright\arraybackslash}p{0.2 * linewidth}}
\caption{\label{tab:unnamed-chunk-1}Evolución del artículo 1 de la CPEUM a
                               partir de 1857.}\\
\hline
\textbf{Const. de 1857\textsuperscript{1}} & \textbf{Const. de 1917\textsuperscript{2}} & \textbf{Ref. de 2001\textsuperscript{3}} & \textbf{Ref. de 2006\textsuperscript{4}} & \textbf{Ref. de DDHH de 2011\textsuperscript{5}}\\
\hline
\endfirsthead
\caption[]{Evolución del artículo 1 de la CPEU \textit{(continued)}}\\
\hline
\textbf{Const. de 1857\textsuperscript{1}} & \textbf{Const. de 1917\textsuperscript{2}} & \textbf{Ref. de 2001\textsuperscript{3}} & \textbf{Ref. de 2006\textsuperscript{4}} & \textbf{Ref. de DDHH de 2011\textsuperscript{5}}\\
\hline
\endhead
El pueblo mexicano reconoce que los derechos del hombre son la base y el objeto de las instituciones sociales. En consecuencia declara que todas las leyes y todas las autoridades del pais deben respetar y  sostener las garantías que otorga la presente Constitucion. & \textbf{En los Estados Unidos Mexicanos todo individuo gozará de las garantías que otorga esta Constitución, las cuales no podrán restringirse, sino en los casos y con las condiciones que ella misma establece.} & En los Estados Unidos Mexicanos todo individuo gozará de las garantías que otorga esta Constitución, las cuales no podrán restringirse, sino en los casos y con las condiciones que ella misma establece.\textbf{\textsuperscript{(sin cambio)}} & En los Estados Unidos Mexicanos todo individuo gozará de las garantías que otorga esta Constitución, las cuales no podrán restringirse, sino en los casos y con las condiciones que ella misma establece.\textbf{\textsuperscript{(sin cambio)}} & En los Estados Unidos Mexicanos \textbf{todas las personas gozarán de los derechos humanos reconocidos por esta Constitución y en los tratados internacionales de los que el Estado Mexicano sea parte, así como de las garantías para su protección, cuyo ejercicio no podrá restringirse ni suspenderse, salvo en los casos y bajo las condiciones que esta Constitución establece.}\\
\hline
 &  &  &  & Las normas relativas a los derechos humanos se interpretarán de conformidad con esta Constitución y con los tratados internacionales de la materia favoreciendo en todo tiempo a las personas la protección más amplia.\\
\hline
 &  &  &  & Todas las autoridades, en el ámbito de sus competencias, tienen la obligación de promover, respetar, proteger y garantizar los derechos humanos de conformidad con los principios de universalidad, interdependencia, indivisibilidad y progresividad. En consecuencia, el Estado deberá investigar, sancionar y reparar las violaciones a los derechos humanos, en los términos que establezca la ley.\\
\hline
 &  & Está prohibida la esclavitud en los Estados Unidos Mexicanos. Los esclavos del extranjero que entren al territorio nacional alcanzarán, por este solo hecho, su libertad y la protección de las leyes. & Está prohibida la esclavitud en los Estados Unidos Mexicanos. Los esclavos del extranjero que entren al territorio nacional alcanzarán, por este solo hecho, su libertad y la protección de las leyes.\textbf{\textsuperscript{(sin cambio)}} & Está prohibida la esclavitud en los Estados Unidos Mexicanos. Los esclavos del extranjero que entren al territorio nacional alcanzarán, por este solo hecho, su libertad y la protección de las leyes.\textbf{\textsuperscript{(sin cambio)}}\\
\hline
 &  & Queda prohibida toda discriminación motivada por origen étnico o nacional, el género, la edad, las capacidades diferentes, la condición social, las condiciones de salud, la religión, las opiniones, las preferencias, el estado civil o cualquier otra que atente contra la dignidad humana y tenga por objeto anular o menoscabar los derechos y libertades de las personas. & Queda prohibida toda discriminación motivada por origen étnico o nacional, el género, la edad, las \textbf{discapacidades}, la condición social, las condiciones de salud, la religión, las opiniones, las preferencias, el estado civil o cualquier otra que atente contra la dignidad humana y tenga por objeto anular o menoscabar los derechos y libertades de las personas. & Queda prohibida toda discriminación motivada por origen étnico o nacional, el género, la edad, las discapacidades, la condición social, las condiciones de salud, la religión, las opiniones, las preferencias \textbf{sexuales}, el estado civil o cualquier otra que atente contra la dignidad humana y tenga por objeto anular o menoscabar los derechos y libertades de las personas.\\
\hline
\insertTableNotes
\end{longtable}
\end{ThreePartTable}
\end{landscape}

\end{document}
